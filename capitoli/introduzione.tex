La presente tesi è frutto del lavoro svolto durante il tirocinio curricolare presso STAM Tech S.r.l.,
azienda Genovese operante nel settore IT, e attiva nello svilippo di tecnologie innovative a livello nazionale e internazionale.

\section{Contesto e obiettivi del progetto}

Il progetto è stato pensato per esplorare e implementare soluzioni innovative per l'integrazione di strumenti di pipelining AI alternativi ad Apache Airflow, con l'obiettivo di migliorare l'efficienza e la scalabilità
dei flussi di lavoro di machine learning.
Il cuore del progetto risiede nell'utilizzo di \textbf{KubeFlow},e del suo modulo \textbf{Pipelines} per l' escuzione e monitoraggio del processo di training dei modelli.
\\

Gli obiettivi principali del progetto includono:
\begin{itemize}
    \item Studio e comprensione delle funzionalità di KubeFlow e della sua integrazione con Kubernetes.
    \item Sviluppo di pipeline di machine learning utilizzando KubeFlow Pipelines, prendendo come base il modello Faster R-CNN.\@
    \item Studio dello strumento \textbf{MLFlow} come strumento di raccolta e monitoraggio delle metriche di training dei modelli.
    \item Valutazione dell' efficacia di KubeFlow rispetto ad Apache Airflow in termini di scalabilità, facilità d'uso e di come l' integrazione
    di MLFlow possa migliorare il monitoraggio dei modelli in fase di training.
\end{itemize}

Il lavoro svolto ha richiesto uno studio approfondito delle tecnologie coinvolte, nonchè una fase di sperimentazione pratica su come allenare un modello di object detection,
e uno studio di come monitorare le metriche di training in modo efficace.